\documentclass[12pt,a4paper,twoside,openright,BCOR10mm,headsepline,titlepage,abstracton,chapterprefix,final]{scrreprt}

\usepackage{ae}
\usepackage[ngerman, english]{babel}
%\usepackage{SIunits}

\usepackage{amsmath}
\usepackage{amssymb}
\usepackage{amsfonts}
\usepackage{xcolor}
\usepackage{setspace}

%opening
\title{Issues to be adressed in Pyrate}
\author{Moritz Esslinger, Johannes Hartung, Uwe Lippmann}

\begin{document}

\maketitle

\begin{enumerate}
 \item minor: use private variables {\tt \underline{\;}\,\underline{\;}variable} and {\tt @property} and {\tt @variable.setter} getter and setter decorators to
      achieve a controled access to private variables but access them in a manner one would do for public variables (i.e. {\tt object.variable})
 \item Check for algorithms and formulas which are only valid in rotational symmetric systems
 \item Check for algorithms that are valid without immersion
 \item Optical system has to calculate absolute positions of its surfaces (including tilt) referenced to a given coordinate system.
 \item Optical system has to calculate relative coordinate transforms (translation, rotation) between certain elements
 \item Calculate coordinate transforms from ABCD/XYUV formalism
 \item Calculate XYUV matrices as generalization of ABCD
 \item Use some pilot ray (in optical axis system) for calculating (exact) intersection points at each surface, calculate local curvature at intersection point
      and calculate focal power in $x, y$. Calculate also tilt (and maybe decenter) of pilot ray at each surface. Combine both results as a matrix product in a resulting
      XYUV matrix
 \item Consider whether XYUV matrices should be 4x4, 5x5, 6x6, or 7x7.
 \item Raybundle.o array is referenced relative to the next surface vertex $\Rightarrow$ counter intuitive!
 \item Use absolute coordinates for 2D and 3D drawing procedures
 \item every class one file
 \item Plots of typical rayfans, wave aberration, Seidel coefficients, aberration plots vs pupil or field coordinates, PSF, image simulation, MTF, OTF
 \item ZMX import
 \item ZMX export
 \item GRIN (propagate function also for other materials)
 \item anisotropic media (uniaxial [partly in rtfm], biaxial)
 \item Pupil aiming for infinite object/image distance
 \item issue: getAllOptimizableVariables in opticalsystem?
 \item long term goal: investigate class structure to be simplified by design concepts of gang of four (Gamma, Helm, Johnson, Vlissides)
 \item long term goal: interface for freecad (systematically implement functionality)
 \item long term goal: Optimization: discrete for glass types, genetic/evolutionary algorithms
 \item long term goal: Image System data base and paraxial layout wizard
 \item long term goal: polarization
 \item long term goal: Fresnel coefficients, generalized for anisotropic media
\end{enumerate}


\end{document}
